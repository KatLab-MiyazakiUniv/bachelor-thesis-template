%%
% 今時jarticleやjbook使ってる人いる?時代はjsarticleかjsbookだよ
% ついでに言うと、uplatexってのはplatexの上位互換、これを使わないなんて旧世代だよね
%
\documentclass[uplatex, report, a4j, 10pt]{jsbook}


%%
% パッケージ群
%
\usepackage{packages/miyazaki-u-paper}   % 宮崎大学工学部の卒論の基本(片山先生作)を、僕がちょっと書き換えちゃった(テヘッ
\usepackage{enumitem}           % enumerate?古い古い
\usepackage[dvipdfmx]{graphicx} % 当然dvipdfmなんて使ってないよね
\usepackage[dvipdfmx]{color}    % listingsを使うときにはこれも必須、dvipdfmxを変えちゃうとgraphicxのdvipdfmxも変わるよ
\usepackage{listings, packages/jlisting} % コードを埋め込むなら必須
\usepackage{txfonts}            % フォントといえばやっぱりtxfonts、今はnewtxってのもあるらしい
\usepackage{verbatim}           % コメントアウトしてくれる便利なプリアンブルが使える \begin{comment} ... \end{comment}
\usepackage{url}
% \usepackage{easy-todo}
\usepackage[hdivide={21mm, , 21mm}, vdivide={30mm, , 25mm}]{geometry} % スタイルを少し変えたくても\hoffset, \voffsetは使わないでね
\usepackage{multirow}
\usepackage{ascmac}


% \usepackage{latexsym}
% \usepackage{bmpsize}
% \usepackage{comment}

% \def\Underline{\setbox0\hbox\bgroup\let\\\endUnderline}
% \def\endUnderline{\vphantom{y}\egroup\smash{\underline{\box0}}\\}

\newcommand{\ttt}[1]{\texttt{#1}}

%%
% miyazaki-u-paper.sty用設定値
%
\degree{g} % Graduateのg or Masterのm
\figurenumbering{f} % 図目次を付ける場合はt (真) を持つ真偽値を引数に取る関数
\tablenumbering{f} % 表目次を付ける場合はt (真) を持つ真偽値を引数に取る関数
\title{Sample Title}
\studentNumber{学籍番号} % 修論では無視する
\author{著者名}
\nendo{2} % 年度
\advisor{片山 徹郎 教授} % 修論では無視する
\major{工学科 情報通信工学プログラム}
\submitDate{提出日} % デフォルトは編集した日付


\begin{document}
\maketitle

%
% 概要
% 
\preface{概要}
ここに概要を書く。


%
% 本文
% 
\input{chapters/10-introduction.tex}
\chapter{研究の準備}\label{cha:Preparation}

Coming soon...

\chapter{MATCHの外観と機能}\label{cha:Function}

Coming soon...

\chapter{MATCHの実装}\label{cha:Implementation}

Coming soon...

\chapter{適用例}\label{cha:Indication}

Coming soon...

\chapter{考察}\label{cha:Discussion}

Coming soon...

\chapter{おわりに}\label{cha:Conclusion}

Coming soon...


%%
% 謝辞
%
\acknowledgment

Coming soon...


%%
%参考文献
%
\begin{thebibliography}{0}
	\bibitem{example}example\\
	\url{http://example.com}\\アクセス日:1970/01/01.
\end{thebibliography}

% \newpage
% \listoftodos
\end{document}
