\documentclass[uplatex]{jsarticle}

% 1段組
\usepackage{AbstractStyle}

% 2段組
% \usepackage[twocol]{AbstractStyle}

\begin{document}

\setAbstractTitle{論文タイトルだよ}
\setAbstractPresenter{発表者名だよ}

\AbstractHeader

\begin{AbstractBody}
スマートフォンやタブレット向けのアプリケーション市場は年々拡大しており、モバイルアプリの大規模化と複雑化が進んでいる。

大規模化と複雑化が進んだモバイルアプリは、複数のアーキテクチャを混合して実装している場合がある。
例えば、ある画面と機能はMVVMアーキテクチャに、別のある画面と機能はレイヤーアーキテクチャに準拠して実装し、新しく追加する画面と機能はThe Composable Architecture(TCA)に準拠して実装する場合がある。
そのようなモバイルアプリの構造を理解するためには時間がかかり、開発者がコードに変更を加える際に、既存の実装を理解したり、その変更の影響範囲を把握したりしながら設計することは困難である。
影響範囲を把握せずに変更を加えると、意図していない箇所で不具合が発生してしまう可能性がある。

ソースコード中の依存関係理解を支援する機能として、Java向けの統合開発環境Eclipseでは、継承関係を表すクラスヒエラルキーや、関数呼び出しを表すコールヒエラルキーを表示する拡張機能を利用できる。
しかし、プログラミング言語Swiftを使用するiOSアプリは、Xcodeでの開発に限られている。
開発環境の制約により、iOSアプリ開発において、JavaやC++での開発における理解支援機能を使用することは難しい。

以上から、大規模化と複雑化、開発環境の制約により、Swiftを使用するiOSアプリ開発では変更の影響範囲を把握しながら設計することが困難であるという問題がある。
この問題をソフトウェア可視化により解決し、開発者の設計と影響範囲理解を支援するために、SwiftDiagram を提案した。
SwiftDiagram は、Swiftで記述されたソースコードの静的構造と影響範囲を可視化する図である。
しかし、SwiftDiagram を手動で描画することは手間がかかり、人手によるミスが発生する可能性がある。

そこで本研究では、プログラミング言語SwiftによるiOSアプリ開発の支援を目的として、SwiftDiagram を自動生成するツールRAGESS(Real-time Automatic Generation SwiftDiagram System)) を実装した。
RAGESS は、プロジェクトを監視し、ビルド開始を検知するとプロジェクト内のSwiftソースコードを抽出する。
そして、ビルド成功を検知すると、ソースコード中のオブジェクトの定義を抽出する。
定義の抽出は、Swift Syntaxを使用してソースコードを抽象構文木に変換し、その抽象構文木を走査し、オブジェクトの定義に対応するノードに到達した際にインスタンスを生成することで実現する。
定義の抽出後、ビルド時に生成されたDerivedDataディレクトリ中のIndexStoreを解析し、依存関係を抽出する。
ビルド開始検知時に抽出したソースコードをビルド成功検知時に解析することで、シンタックスエラーを含まないソースコードを解析することを保証する。
解析後、ファイルツリーエリアで目的の型を選択することで、その型の影響範囲を表すSwiftDiagram を表示する。
RAGESS は、ビルド成功を検知する度に解析を自動で行い、SwiftDiagram を更新する。

実装したRAGESS のリアルタイム性を、RAGESS 自体の約14,000行のSwiftソースコードを適用して評価した。
RAGESS のビルドにかかった時間である2分11秒に対して、RAGESS は、ビルドの開始検知から1秒未満でソースコードを抽出し終えた。
そして、ビルドの成功検知から3秒後に、ソースコードからオブジェクトの定義を抽出し終えた。
さらに、定義を抽出し終えて1秒未満で依存関係を抽出し終えた。
また、ファイルツリーエリアで影響範囲が広い型を選択した際、10秒後に、その型の影響範囲を表すSwiftDiagram を表示した。
ビルドにかかった時間と比較して、RAGESS の処理時間は短く、リアルタイム性があると評価した。

最後に、RAGESS と関連ツールを比較した。
まず、Swiftソースコードを対象とした可視化ツールであるEmergeおよびSwiftcityと比較した。
2つのツールと比較して、RAGESS は、SwiftDiagram で可視化することで、型や構成要素の静的構造と、よりSwiftの構文規則に合わせた影響範囲を把握しながら設計することを支援できる。
また、RAGESS は、解析した情報を外部ファイルに出力したり、Webサイトにアップロードしたりする必要がないため、セキュリティを重視する開発でも使用できる。
次に、AIによる解析機能を持つエディタであるCursorと比較した。
その結果、RAGESS は、構造体の影響範囲について、マクロで追加される要素による参照や、イニシャライザの処理中での参照も抽出し、影響範囲を正しく可視化できた。

以上より、本研究で実装したRAGESS は、iOSアプリ開発を支援し、iOSアプリの大規模化と複雑化、開発環境の制約により変更の影響範囲を把握しながら設計することが困難であるという問題を解決することに有用であると言える。

\end{AbstractBody}

\end{document}
